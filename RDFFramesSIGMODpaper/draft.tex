

\sparql is a data extraction langage and dataframes are a data analysis tools. We are bridging the gap between the two. 

Providing formal semantics for a query language is a challenging task. For \sparql, we have the W3C Recommendation that is written in natural language which is inherently ambiguous. This has resulted in different implementations in different query engines and made it confusing for users who try to learn it and use it to query their RDF data. In this project, we need to learn the semantics of \sparql to derive the equivalence between our API calls and \sparql queries we build automatically and justify the optimizations we made. There has been few attempts to provide semantics for \sparql but most of them made simplifying assumptions like omitting the bag semantics % 3 papers as examples here
or omitting the groupby and aggregations. % do they assume the absence of null values?

Using the bag semantics preserves the data distribution which is of paramount importance to machine learning and data science applications.\cite{sqlsemantics}

Given how most of knowledge graphs are constructed, they are far from being complete. Allowing optional patterns is an essential featute of \sparql that we use. 

\subsection{Data frames}

In this section we describe the data model of the tables generated by the API. We use the tidy data philosophy  proposed in \cite{tidydata} for structuring datasets to facilitate analysis which has been realized in the DataFrame data structure in in the main languages used for data science like python and R.
The data frame is a standard tabular data structure that was designed to facilitate initial analysis of data and development of analysis and machine learning tools. It stores tabular data as a list of vectors of the same length with integrated handling of metadata \cite{pandas} -rows and coloumns labels- and make statistical analysis, cleaning and visualization of tabular data easy and efficient. 

The principles of tidy data are closely tied to those of relational databases and relational algebra \cite{relalgebra} but are framed in procedural APIs in languages familiar to data scientists. Conventionally, dataframes are created from the result of a SQL query on relational data. It provides a tidy representation of data that can be integrated with data analysis and machine learning tools easily and make data cleaning and analysis efficient.

One of the main challenges of analyzing data in the RDF model is the time and effort required to import it to the tidy data tools. In this API, we fill this gap by creating a dataframe from the result of a \sparql query on RDF data. We show that every table returned by our API can be stored in a dataframe.

% define a dataframe, charecteristics of a dataframe
A Dataframe is a table or a two-dimensional array-like structure describing one type (class) of entities in which each column represents a variable and contains values that measure some underlying attribute and each row represents an observation which is the value of all the observed variables (attributes) for one instance. Each type of instances is stored in a different class.  The following properties hold for our dataframes:
\begin{itemize}
    %\item Each variable forms a column.
    %\item Each observation forms a row.
    %\item Each type of observational unit forms a table.
    \item The column names should be non-empty.
    \item The column names should be unique. ??
    \item The row names should be unique. ?? Do we have row names? Do rows need to be unique? No
    \item The data stored in a data frame can be of numeric, factor or character type. Are these the only allowed types?
    \item Each column should contain same number of data items.?? including NULL?
    \item If you have multiple tables, they should include a column in the table that allows them to be linked.
\end{itemize}{}

Operations on Dataframes:
\begin{itemize}
    \item create a dataframe form the result of a \sparql query.
    
\end{itemize}{}
%\subsection{Relational Data Model}
%a tuple is imply a sequence (or list) of values. A relationship between n values is represented mathematically by an n-tuple of values, i.e., a tuple with n values, which corresponds to a row in a table. Thus, in the relational model the term \textbf{relation} is used to refer to a table, while the term \textbf{tuple} is used to refer to a row. Similarly, the term \textbf{attribute} refers to a column of a table. We use the term relation instance to refer to a specific instance of a relation,i.e., containing a specific set of rows.
%The following chercteristics hold for relational data:
%\begin{itemize}
%    \item for all relations r, the domains of all %attributes of r be atomic. (if elements of the domain %are considered to be indivisible units)
    %\item The null value is a special value that signifies %that the value is unknown or does not exist.
%\end{itemize}{}

%Relational Operations:
%a set of operations that can be applied to either a single relation or a pair of relations. These operations have the nice and desired property that their result is always a single relation.
%\begin{itemize}
%    \item selection
%    \item projection
%    \item inner join
%    \item Cartesian product
%    \item union 
%\end{itemize}{}